\documentclass{article}
\usepackage[T1]{fontenc}
\usepackage[utf8]{inputenc}
\usepackage[margin=1in]{geometry}
\usepackage[bookmarks=true,pdfstartview=FitH,bookmarksopen=true]{hyperref}
\usepackage{bookmark}
\usepackage{url}

\newcommand{\HRule}{\rule{\linewidth}{0.5mm}}
\newcommand{\Hrule}{\rule{\linewidth}{0.3mm}}

\makeatletter% since there's an at-sign (@) in the command name
\renewcommand{\@maketitle}{%
  \parindent=0pt% don't indent paragraphs in the title block
  \centering
  {\Large \bfseries{\@title}}
  \HRule\par%
  \textit{\@author}
  \par
}
\makeatother% resets the meaning of the at-sign (@)

\title{Statement of Purpose}
\author{}

\begin{document}
  \maketitle % prints the title block

Building small computer games, line follower robots and repairing computers, I got overwhelmed with electronics and programming since my school days. I was exceptional in maths and computers. So, I decided to make a career in the software industry.\\

After completing my under-graduation, I got selected in a product development company as a Software developer from the campus interview. I got the opportunity to work on Linux platform for the first time. So, I learned about Linux by reading books and watching tutorials. From that point onwards, I kept on learning about new computer science concepts and technologies by myself. By working on real world projects in a team environment, I discovered new skills such as teamwork, planning and time management.\\

I have learned about core computer science subjects such as Algorithms, Data structures and Operating systems on my own. I have  also participated in online competitive programming contests on HackerRank to improve my programming skills. I educate myself by taking different online courses on edx, coursera and udacity. I do freelancing in my spare time to apply my skills which I gained from online courses into the real world projects. I regularly do contribution in open source projects hosted by \href{http://www.GitHub.com/mehul-m-prajapati}{GitHub} with other contributors to make an impact in a large community.\\

I attend technological meet-ups and events organized from \href{www.meetup.com}{meetup.com} to update myself with latest technologies. The renowned groups in which I go are as follows.

\begin{itemize}
   \item \href{https://developers.google.com/groups/chapter/106261089114347152720/}{Google Developer Groups}
   
   \item \href{http://mozillaindia.org/}{Mozilla} 
   
   \item \href{https://www.meetup.com/ahmedabad-wp-meetup/members/72560962/}{Wordpress}
   
   \item \href{https://pythonexpress.in/}{Python Express}
\end{itemize}

I like solving challenging problems. The kind of problems that have not been solved before, that might not even appear to be problems at first sight. The kind of problems one learns to solve at research universities.\\

My research interest lies in Machine Learning. I have recently developed a deep learning based chatbot leveraging TensorFlow library. This project tries to reproduce the results of a Neural Conversational Model. It uses a RNN (seq2seq model) for sentence predictions. It is done using python and TensorFlow. I am currently doing some research in deep learning on my own.\\

My long term goal is to start my own organization which develops machine learning based innovative products. I have the work experience, the motivation, and the entrepreneurial spirit. I want to achieve expertise in machine learning by going through the graduate level program. I believe University of Toronto’s Applied Computing graduate program would be the best place for me to gain that.\\

I want to study at the University of Toronto for a number of reasons; first and foremost is that there are potential supervisors whose research interests aligned well with my own. Secondly, the University of Toronto is a global leader in the research. It has many interesting areas of study within my field of interest and a beautiful campus. Lastly, I will get the opportunity to contribute in the university community by which I will feel proud.  
  \end{document}